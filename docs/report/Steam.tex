\section{湯気のシミュレーション}

\subsection{湯気の発生と消滅のプロセス}
\begin{figure}[h]
\centering
\includegraphics[width=0.8\linewidth]{process.eps}
\caption{湯気の発生と消滅のプロセス}
\label{process}
\end{figure}
湯気が発生してから消滅するまでのプロセスを図(\ref{process})に記載する.	
初期状態としては常温の空気内に沸騰状態ではないが空気より高い温度の水があると仮定する.
水は水面上から蒸発し,蒸発した水蒸気は分子拡散により周囲へ拡散する.
同時に水の温度は空気よりも高いため温度勾配により熱拡散が起こり熱が水から空気へ移動する.
空気の温度が上昇すると空気の密度が周囲の空気より低くなることにより浮力が発生する.
浮力により空気に上方向の力が働き,対流が発生する.
この対流により空気に含まれる水蒸気,熱が移動する.
空気中に含まれる水蒸気量が飽和水蒸気量を超えた場合に凝結することで水滴が発生する.
凝結は大気中の細かい塵を核として行われる.
この水滴の粒子に対して光が当たった際に散乱が起こり白く見えることで湯気として認識される.
凝結により発生する水滴の粒子は光の波長と同程度,もしくは少し大きめの大きさとなりミー散乱という現象が発生する.
ミー散乱は粒子のサイズが大きくなるにつれて前方の指向性が高くなり側方,後方への散乱は弱くなる.
空気中に含まれる水蒸気量が飽和水蒸気量を下回った場合,水滴が空気中に蒸発し湯気が消滅する.

\subsection{シミュレーションモデル}

流体の速度$\upsilon=(u,v,w)$は非粘性,非圧縮のオイラーの運動方程式(\ref{continuity},\ref{euler})によって与えられる.

\begin{equation}
\label{continuity}
\nabla \cdot \upsilon = 0
\end{equation}
\begin{equation}
\label{euler}
\frac{\partial \upsilon}{\partial t} = -(\upsilon \cdot \nabla)\upsilon - \nabla p + B + f
\end{equation}

$p$は圧力,$B$は浮力,$f$は風などによる外力を表す.浮力$B$はブシネスク近似より式(\ref{buoyancy})で定義する.
\begin{equation}
\label{buoyancy}
B=k_{b}\frac{T-T_{a}}{T_{a}}z
\end{equation}
$k_{b}$は浮力の係数,$T$は流体の温度,$T_{a}$は環境温度,$z$は上方向のベクトルである.

湯気の密度$q_{c}$と水蒸気$q_{v}$の密度は次式で定義する.
\begin{equation}
\label{steam}
\frac{\partial q_{c}}{\partial t} = -(\upsilon \cdot \nabla)q_{s} + C_{s}
\end{equation}
\begin{equation}
\label{vapor}
\frac{\partial q_{v}}{\partial t} = -(\upsilon \cdot \nabla)q_{v} + D_{v}\nabla^2q_{v} - C_{s} + S_{v}
\end{equation}
$D_{v}$は水蒸気の分子拡散係数,$C_{s}$は相転移によって発生する湯気の量,$S_{v}$は水蒸気源から水蒸気の供給量である.

温度$T$は次式で表される.
\begin{equation}
\label{temperature}
\frac{\partial T}{\partial t} = - (\upsilon \cdot \nabla)T + D_{t}\nabla^2T +  QC_{s} + S_{T}
\end{equation}
ここで$D_{t}$は熱拡散率,$Q$は潜熱係数を表す.
右辺第一項は熱対流,第二項は熱拡散,第三項は相転移による潜熱,第四項は外部の熱源からの熱量を表す.

分子拡散係数$D_{v}$は次式で表される.これはアインシュタイン・ストークスの式より温度に依存する.
\begin{equation}
\label{diffusion}
D_{v}=D_{0}T
\end{equation}
ここで$D_{0}$は分子拡散係数を決定するためのパラメータである.

相転移によって増減する湯気の量$C_{s}$は次式で表される.
\begin{equation}
\label{transition} 
C_{s} =
\begin{cases}
 \alpha(q_{v}-q_{s}) & q_{v} \geq q_{s}\\
 \max\left(\alpha(q_{v}-q_{s}),-q_{c}\right) & q_{v} < q_{s}
\end{cases}
\end{equation}
\begin{equation}
\label{saturation}
q_{s} = \min\left(S_{a} \exp\left(\frac{-S_{b}}{T+S_{s}}\right),q_{v}+q_{c}\right)
\end{equation}
ここで$\alpha$は相転移率である.
$q_{s}$の$\min$関数の第一引数は飽和水蒸気密度を表し
$S_{a},S_{b},S_{s}$は飽和水蒸気密度を決定するためのパラメータである.
$q_{s}$の$\min$関数の第二引数で湯気と水蒸気の密度の合計値を指定する.
これにより水蒸気と湯気の密度の合計値が飽和水蒸気密度を超えている場合にも湯気の消滅を行う.
湯気の消滅は湯気の密度以上行うことはない.

湯気の速度$\upsilon_{s}$は流体との相対速度から発生する抗力,重力を考慮して次式で表される.

\begin{equation}
\label{lagurange}
\frac{d\upsilon_{s}}{dt} = - F_{drag} + mg
\end{equation}
\begin{equation}
\label{dragforce}
F_{drag} = - C_{D} (\upsilon_{s} - \upsilon)^\beta
\end{equation}
$F_{drag}$は抗力,$C_{D}$は抗力を決定するためのパラメータ,$\beta$はレイノルズ数を表しこれは粒子の半径に依存する.

\subsection{実装}

\begin{figure}[h]
\includegraphics[width=0.7\linewidth]{simulation.eps}
\caption{湯気のシミュレーション空間}
\label{simulation}
\end{figure}

\begin{figure*}[h]
\centering
\includegraphics[width=0.8\linewidth]{overview.eps}
\caption{提案法の概要}
\label{overview}
\end{figure*}

シミュレーション空間は$N_{x} \times N_{y} \times N_{z}$の格子に分割し各格子点に温度$T$,水蒸気の密度$q_{v}$,流体の速度$\upsilon$を割り付ける.格子には温度,水蒸気の密度を格子の面に定義するスタガード格子を採用する.
初期状態として水蒸気源と熱源が存在する部分には水蒸気密度と温度の固定値を割り付ける.
水蒸気源と熱源は空間の底面から発生し,発生量の分布はユーザにより定義する.
湯気は粒子により表現し粒子には格子空間上の位置,速度,質量が格納される.
粒子は質量が一定で粒子同士の衝突,粒子から格子の流体の速度,温度に対する影響はないと仮定する.

図\ref{overview}に本シミュレーション空間上での1タイムステップ中の処理の流れを示す.

\begin{enumerate}
\item {\bf 粒子を格子の値へ変換.}格子内の湯気の粒子の質量を合計し,すべての格子に対して湯気の密度$q_{c}$を計算する.
\item {\bf 相転移による粒子,格子の値の追加,削除.}式(\ref{transition})の相転移のモデルを用いて相転移によって増減する湯気の量$C_{s}$を求め,湯気と水蒸気量の追加,削除を行う.
$C_{s} \geq 0$の場合,格子内に湯気の粒子を追加する.
湯気の粒子の位置は格子内のランダムの位置とし,速度は粒子の位置にある速度を格子面に定義される速度から線形補間により求め,質量は一定とする.
これを追加した粒子の質量の合計が$C_{s}$になるまで処理を続ける.
$C_{s} < 0$の場合,格子内の湯気の粒子の質量を削除する.
これを削除した粒子の質量の合計が$C_{s}$になるまで処理する.
湯気の粒子の追加と削除もしくは処理の後,温度へ$C_{s}$に依存した潜熱の追加,水蒸気の密度$q_{v}$から$C_{s}$の減算,質量が0の湯気の粒子の削除処理を行う.
\item {\bf 格子の速度計算.}非粘性,非圧縮のオイラーの運動方程式(\ref{continuity},\ref{euler})をFedkiwら\cite{Fedkiw2001}の手法に基づき外力,浮力,圧力,移流により流体の速度を計算する.
\item {\bf 格子の値更新.}温度と水蒸気量を式(\ref{temperature},\ref{vapor})に基づき計算する.式(\ref{temperature},\ref{vapor})は共に第一項が対流項,第二項が拡散項となる.
対流項はセミラグランジュ法,拡散項は拡散方程式の数値解析を用いる.拡散項の数値解法で本研究で採用した陽解法の場合はCFL条件によりタイムステップ幅に厳しい制限を課す必要がある.
\item {\bf 粒子の速度計算.}粒子の速度を式(\ref{lagurange})に基づき計算する.抗力$F_{drag}$は格子と粒子の速度の間の相対速度により求める.この計算で用いる格子の速度は粒子の位置にある速度を格子面に定義される速度から線形補間により求める.
\item {\bf 粒子の位置更新.}粒子の位置を粒子自体の速度を追加することにより求める.
\end{enumerate}

