\section{今後の研究計画}

\subsection{研究の方針}

本論文で提案したモデルの場合、湯気らしい映像を得ることができたがパラメータにより得られる映像がかなり異なる。
そのため湯気らしいパラメータについて調査と調整を行う。
レンダリングには簡易的な方法を用いているためPOV-Ray,PBRT,Mitsubaなどのレンダリングソフトウエアを用いてレンダリングを行う。
これによりリアリティのある映像を得ることができると考えられる。
また拡散方程式の解法として現在は陽解法を用いているが、不安定になる場合があるため陰解法による解法にも対応を行う。
上記が完了した場合、下記に記載の発展的研究も実施する。

\subsection{発展的研究}
計画に記載した湯気のモデル化手法は10年以上前の古典的な手法を組み合わせ、改変したものである。現在では湯気に関連のある新たな手法があり、これらを利用することでよりリアリティのある表現が可能となることが考えられる。湯気の発生源のシミュレーションと煙の乱流のシミュレーションで用いる渦度強制法(Vorticity Confinement)にウェーブレットノイズを用いた手法\cite{Kim2008}を利用することでリアルな湯気の発生と高解像度な乱流を生成することができる。また湯気の発生と消滅のモデルは雲のモデルの利用を予定しており、スケールが異なるため十分な結果を得られない可能性がある。そのため流体力学の混相流のシミュレーションの分野で用いられる手法を取り入れることも考えられる。例えば、\cite{Mizuno2003}では火山噴煙のシミュレーションに2流体モデルを利用している。水滴の表現としては水煙の表現で用いられる手法\cite{Nielsen2013}の利用が検討できる。
また湯気の発生箇所としては水分が含まれる液体状の流体であるシーンが多い。
液体状の流体から発生する湯気をシミュレートすることにより、さらに表現力を高められると考えられる。
湯気の発生元のシミュレーションはパーリンノイズを利用することを想定しているが、柔軟に流体を表現することができるFLIP法\cite{Zhu2005}を用いることも検討したい。

湯気の表現としては現実を再現した場合に必ずしも効果的な湯気の表現とならない。例えば、CGを利用せずに湯気を含む写真の撮影する際、ドライアイス等で外気の温度を下げることで効果的な湯気を生成する技法がある。このように効果的な湯気を生成するためにユーザが自由に湯気をデザインできるモデリングツールの開発が応用例として考えられる。
また、GPUを利用した高速化・リアルタイム化、バーチャルリアリティ技術への適用、湯気だけでなく物体へのしみ込みや凝結といった水のさらなる表現力の向上に取り組むことも考えられる。

\subsection{評価方法}
本研究は過去の研究事例が無い状況であるため過去事例との比較が困難となっている。
CGは映画などの映像に利用されることを目的としているため、見た目として正しく表現されていることが求められ、必ずしも物理的に正しいことは求められない。
そのためCGにおいて新規の表現手法を提案する場合の評価として定性的なものとなり定量的な測定は困難である。

実際の研究事例から新規の表現手法にどのような評価手法が取られているかを確認した。
CGにおいて初めて水\cite{Foster1996}や雲\cite{Gardner1985}がシミュレートされた例では出力された絵とその説明のみとなっている。
火山噴煙\cite{Mizuno2003}やプロミネンス\cite{Ishikawa}が初めてシュミレートされた場合は実際の写真と出力された絵を並べることにより比較を行っている。

上記より評価の手法としては定量的な評価ではなく、出力された絵の説明を行い、可能であれば実際の写真との比較により行うこととする。
