\section{はじめに}
本書では2016年8月時点の研究活動の中間報告を行う.

\subsection{背景・目的}
近年、3Dプリンタ、プロジェクションマッピングやバーチャルリアリティ技術の普及より、コンピュータグラフィックス(CG)は重要な技術になっている.
これらの3D技術が身近になることで、現実世界に近いリアルな画像やアニメーションの作成技術もさらに重要になると予想される. CGの研究においては水の表現のような流体シミュレーションに関する研究が盛んに行われているが水の特徴である三態変化を扱う研究はほとんど行われていない.
特に水が液体から気体へ相転移する際に発生する、湯気の表現手法は確立していない.
このためCGの分野において湯気の表現はクリエータによる熟練した技術と経験が必要であった.

本研究ではCGによる湯気のシミュレーションと可視化を行うことを目的とする.
具体的には温度、水蒸気量といった湯気の状態に影響するパラメータを変化させることにより容易にCGによる湯気の表現を可能とする.
これにより従来のCGでは作成できない温度、水蒸気量を考慮した温泉、暖かい料理といったシーンを表現可能となり、よりCGの表現能力を上げることができる.なお本研究では常温の空気中に存在する沸騰の状態ではない液体の水から湯気が発生している状態を対象とする.

\section{既存研究}

湯気の発生に関する表現に特化したCGによる流体シミュレーションの研究は行われていない.
湯気に近い表現としては煙\cite{Fedkiw2001},雲\cite{Dobashi2000}\cite{Miyazaki2001}\cite{Miyazaki2002},火山噴煙\cite{Mizuno2003}\cite{Mizuno2004},水煙\cite{Nielsen2013},水滴と水泡\cite{Mihalef2009}のシミュレーションの研究がある.
水の蒸発,凝固,凝結といった性質を包括的に扱う研究\cite{Fujisawa2008}はあるが湯気に関しては考慮されない.
\cite{Foster1997}では湯気(Steam)の表現の言及はあるが水蒸気量を考慮した表現は行われていない.
流体力学では複数の相が混じり合う物質は混相流(multi phase flow),湯気のように気体と液体が混合する物体の動きは気液二相流として研究が行われているがCGの作成を目的とはしていない.

湯気の表現に必要な水の液体と気体間の相転移,温度の熱拡散,蒸気の分子拡散をCGに適用した研究がある.
水の液体と気体間の相転移のモデルは宮崎ら\cite{Miyazaki2001}\cite{Miyazaki2002}により飽和水蒸気量を利用するモデルが提案された.
温度の熱拡散のモデルはFosterら\cite{Foster1997},宮崎ら\cite{Miyazaki2002},蒸気の分子拡散のモデルは宮崎ら\cite{Miyazaki2002}によりCGに適用された.

本研究では湯気の流体現象の計算に粒子法と格子法の両方を用いるFLIP法をベースとした手法を採用する.
CGにおいて粒子法と格子法を組み合わせることによる手法は雨氷\cite{Ishikawa2015},雪\cite{Stomakhin2013}の表現において適用される.

\section{提案法}
本論文では湯気のシミュレーションと可視化を実際の湯気の発生と消滅のプロセスをもとに格子法と粒子法を用いた手法を提案する.
流体の空気と水蒸気の気体部分は格子法,湯気の水滴部分は粒子法によって表現を行い,
流体との相対速度から発生する抗力を考慮した湯気の動きを再現する.
流体のモデルはFedkiwら\cite{Fedkiw2001}の手法に基づき非粘性,非圧縮の流体を仮定する.
熱,水蒸気の発生と移動は熱拡散と分子拡散,流体の速度による自然対流による移動の双方を考慮する.
水滴から水蒸気の間の相転移は宮崎ら\cite{Miyazaki2001}\cite{Miyazaki2002}による飽和水蒸気量を利用するモデルを格子法と粒子法を用いる手法へ適用する.
