\chapter{はじめに}
\section{背景}
近年,3Dプリンタ,プロジェクションマッピングやバーチャルリアリティ技術の普及より,コンピュータグラフィックス(CG)は重要な技術になっている.
これらの3D技術が身近になることで,現実世界に近いリアルな画像やアニメーションの作成技術もさらに重要になると予想される.
特に水の流体表現は流体力学に基づく流体の挙動の計算によるアニメーション作成方法について数多くの研究がされている.
CGの研究においては水の表現のような流体シミュレーションに関する研究が盛んに行われているが水の特徴である三態変化を扱う研究はほとんど行われていない.
特に水が液体から気体へ相転移する際に発生する湯気の表現は,温かい料理,温泉の水面といった我々の日常生活でも触れる機会が多く,CGにおいてはリアリティを高める上で欠かせない要素と考えられる.
またCGの分野においては映画,CM,パンフレットなど湯気を写実的に表現する場面は多々あるが
現実的には湯気の表現は煙の表現をベースとし,クリエータによる熟練した技術と経験が必要となっている.
CGにおいて湯気に特化した表現手法を開発することでCGの表現能力が向上することが見込まれる.

\section{既存研究}

湯気の発生に関する表現に特化したCGによる流体シミュレーションの研究は確認できていない.
湯気に近い表現としては煙\cite{Fedkiw2001},雲\cite{Dobashi2000}\cite{Miyazaki2001}\cite{Miyazaki2002},火山噴煙\cite{Mizuno2003}\cite{Mizuno2004},水煙\cite{Nielsen2013},水滴と水泡\cite{Mihalef2009}のシミュレーションの研究がある.
水の蒸発,凝固,凝結といった性質を包括的に扱う研究\cite{Fujisawa2008}はあるが湯気に関しては考慮されない.
\cite{Foster1997}では湯気(Steam)の表現の言及はあるが水蒸気量を考慮した表現は行われていない.

湯気の表現に必要な水の液体と気体間の相転移,温度の熱拡散,蒸気の分子拡散をCGに適用した研究がある.
水の液体と気体間の相転移のモデルは宮崎ら\cite{Miyazaki2001}\cite{Miyazaki2002}により飽和水蒸気量を利用するモデルが提案された.
温度の熱拡散のモデルはFosterら\cite{Foster1997},宮崎ら\cite{Miyazaki2002},蒸気の分子拡散のモデルは宮崎ら\cite{Miyazaki2002}によりCGに適用された.

CGにおける流体表現の手法としてはオイラー的アプローチによる格子法とラグランジュ的アプローチによる粒子法がある.
格子法はシミュレーション空間を格子状に分割して物理量を離散化する手法でCGにおいてはStamら\cite{Stam1999}の手法がよく知られる.
粒子法は粒子が位置,速度,密度などのパラメータを持ち,近傍の粒子群からカーネル関数により連続的な場を計算する手法でCGにおいてはDesbrunら\cite{desbrun},Mullerら\cite{Muller2003}らのSPHの手法がCGにおいて頻繁に用いられる.
これらの粒子法と格子法を組み合わせることによるFLIP法がBridsonら\cite{Zhu2005}により提案され,雨氷\cite{Ishikawa2015},雪\cite{Stomakhin2013}の表現において適用される.
本論文では湯気の流体現象の計算に格子法と粒子法の両方を用いるFLIP法をベースとした手法を採用する.

\section{本論文の目的と概要}

本論文ではCGによる湯気のシミュレーションと可視化を行うことを目的とする.
なお本研究では常温の空気中に存在する高温の水等の表面から湯気が発生している状態を対象とする.

CGによる物理シミュレーションの場合,物理現象を正確に解くことよりもアーティストがパラメータや操作による変化させることが重要となる.
このため湯気の場合は雲と同様に温度,水蒸気量といったパラメータにより湯気の発生量や形状が変化し,形状変化がユーザーにより操作しやすいことが必要となる.

本論文では湯気のシミュレーションと可視化を実際の湯気の発生と消滅のプロセスをもとに格子法と粒子法を用いた手法を提案する.
流体の空気と水蒸気の気体部分は格子法,湯気の水滴部分は粒子法によって表現を行い,
流体との相対速度から発生する抗力を考慮した湯気の動きを再現する.

流体のモデルはFedkiwら\cite{Fedkiw2001}の手法に基づき非粘性,非圧縮の流体を仮定する.
熱,水蒸気の発生と移動は熱拡散と分子拡散,流体の速度による自然対流による移動の双方を考慮する.
水滴から水蒸気の間の相転移は宮崎ら\cite{Miyazaki2001}\cite{Miyazaki2002}による飽和水蒸気量を利用するモデルを格子法と粒子法を用いる手法へ適用する.

提案した手法により,CGによる湯気の表現を温度,水蒸気量といったパラメータを変化させることにより可能となる.
また水滴部分に粒子法を用いることにより,湯気特有の微細な形状の再現,CGの表現に必要なノイズの追加が容易となる.
これによりCGによる温泉,暖かい料理といった湯気が登場するシーンを再現し,よりCGの表現能力を上げることができる.