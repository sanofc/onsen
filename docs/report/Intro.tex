\section{はじめに}
本書では2016年2月時点の研究活動の中間報告を行う。

\subsection{背景・目的}
近年、3Dプリンタ、プロジェクションマッピングやバーチャルリアリティ技術の普及より、コンピュータグラフィックス(CG)は重要な技術になっている。
これらの3D技術が身近になることで、現実世界に近いリアルな画像やアニメーションの作成技術もさらに重要になると予想される。 CGの研究においては水の表現のような流体シミュレーションに関する研究が盛んに行われているが水の特徴である三態変化を扱う研究はほとんど行われていない。
特に水が液体から気体へ相転移する際に発生する、湯気の表現手法は確立していない。
このためCGの分野において湯気の表現はクリエータによる熟練した技術と経験が必要であった。

本研究ではCGによる湯気のシミュレーションと可視化を行うことを目的とする。
具体的には温度、水蒸気量といった湯気の状態に影響するパラメータを変化させることにより容易にCGによる湯気の表現を可能とする。
これにより従来のCGでは作成できない温度、水蒸気量を考慮した温泉、暖かい料理といったシーンを表現可能となり、よりCGの表現能力を上げることができる。

\subsection{関連研究}
湯気の発生に関する表現に特化したCGによる流体シミュレーションの研究は行われていない。
湯気に近い表現としては煙\cite{Fedkiw2001}、雲\cite{Dobashi2000}\cite{Miyazaki2001}\cite{Miyazaki2002}、火山噴煙\cite{Mizuno2003}\cite{Mizuno2004}、水煙\cite{Nielsen2013}、水滴と水泡\cite{Mihalef2009}のシミュレーションの研究がある。
水の蒸発、凝固、凝結といった性質を包括的に扱う研究\cite{Fujisawa2008}はあるが湯気に関しては考慮されない。
\cite{Foster1997}では湯気(Steam)の表現の言及はあるが水蒸気量を考慮した表現は行われていない。
流体力学では複数の相が混じり合う物質は混相流(multi phase flow)、湯気のように気体と液体が混合する物体の動きは気液二相流として研究が行われているがCGの作成を目的とはしていない。

実用的なCGによる流体シミュレーションはStamによるStableFluid法\cite{Stam1999}により行われた。
これは格子法による差分化を行い、移流項をバックトレースすることにより解くセミラグランジュ法を取り入れた手法で大きな時間幅でも安定的である。
この手法は煙\cite{Fedkiw2001}、水\cite{Foster2001}、火\cite{Nguyen2002}、爆発\cite{Feldman2003}のようなCGによる流体シミュレーションの基礎となった。
StableFluid法は一次の精度のため数値拡散が強く現れる。
Kimら\cite{Kim2005}はBFECC法により高精度な移流を提案した。
BFECC法はバックトレース時にフォワードトレースにより求めた誤差を修正することで	高精度に移流を行う方法である。
Selleら\cite{Selle2008}はMacCormack法によりBFECC法のコストを減らした。
ZhuとBridson\cite{Zhu2005}は格子法とパーティクル法の両方を用いたFLIP法によりさらに数値拡散を減らした。

湯気の表現に必要な水の液体と気体間の相転移、温度の熱拡散、蒸気の分子拡散をCGに適用した研究がある。
水の液体と気体間の相転移のモデルは宮崎ら\cite{Miyazaki2001}\cite{Miyazaki2002}により飽和水蒸気量を利用するモデルが提案された。
温度の熱拡散のモデルはFosterら\cite{Foster1997}、宮崎ら\cite{Miyazaki2002}、蒸気の分子拡散のモデルは宮崎ら\cite{Miyazaki2002}によりCGに適用された。

\subsection{提案手法の概要}
本論文では湯気のシミュレーションと可視化を実際の湯気の発生原理を元にしたモデルと手法を提案する。
気体のモデルはFedkiwら\cite{Fedkiw2001}の手法に基づき非粘性、非圧縮の流体を仮定する。
水の液体と気体間の相転移のモデルは宮崎ら\cite{Miyazaki2001}\cite{Miyazaki2002}による飽和水蒸気量を利用するモデルとする。
対流に必要な浮力はブシネスク近似による誤差を減らすため周囲の温度との差により計算を行う手法を提案する。
熱、水蒸気の移動は流体の速度による移動に加えて、熱拡散、分子拡散を考慮したモデルを提案する。

